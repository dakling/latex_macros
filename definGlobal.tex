% \usepackage{xparse}
\usepackage{yfonts}
\usepackage{amsmath}
\usepackage{amssymb}
\usepackage{csquotes}
\usepackage[noabbrev]{cleveref}
\usepackage{xifthen}

\makeatletter
\newcommand{\MyIfEmptyTF}[1]{%
  \if\relax\detokenize{#1}\relax%
    \expandafter\@firstoftwo%
  \else%
    \expandafter\@secondoftwo%
  \fi
  }

\newcommand*\foreachletter[2]{%
        \begingroup
        \let\templettercommand#1%
        \let\tempspacecommand#2%
        \catcode
        \foreachlettergo
}
\def\foreachlettergo#1{%
        \testletter#1\relax
        \endgroup
}
\def\testletter#1#2\relax{%
        \if#1\otherspace
                \tempspacecommand
        \else
                \templettercommand{#1}%
        \fi
        \ifx\relax#2\relax
                \let\next\relax
        \else
                \let\next\testletter
        \fi
        \next#2\relax
}\catcode`\ 12
\def\otherspace{ }%
\catcode`\ 10
\makeatother

% Default settings for all available options
\newtoggle{EinsteinNotationDiff} % affects second derivatives
\newtoggle{ShortHigherMoments} % abbreviate higher one-point velocity moments
\newtoggle{ShowArgsTwoPoint} % show arguments eg in H_ij(x,y) 
\newtoggle{GroupExpSimplified} % use linear group parameter for scaling symmetries, not exp(a)
\newtoggle{OnePointExponent} % show ^0 for H and R one-point
\newtoggle{EinsteinSumConvention} % Hide sum signs if Einstein summation convention is active

%% general settings
%allow double superscripts
% \catcode`\^ = 13 \def^#1{\sp{#1}{}}
% \catcode`\_ = 13 \def_#1{\sb{#1}{}}

% use with align*-environment
\newcommand\donumber{\addtocounter{equation}{1}\tag{\theequation}}
%% low-level macros
% definition of a good looking bar
\newcommand{\overbar} [ 1 ] {\mkern 1.5mu\overline{\mkern-1.5mu{#1}\mkern-1.5mu}\mkern 1.5mu}
% \newcommand{\overbar} [ 1 ] {\widebar{#1}}

%% semantic macros
% real numbers
\newcommand{\real} [ 1 ] { 
    \MyIfEmptyTF{#1}
        {\mathbb{R} }
        {\mathbb{R}^{#1} }
        }
% symbols to denote entire equations
\newcommand{\nc} [ 1 ] { \mathcal{N}_{#1} }
\newcommand{\ncm} [ 1 ] { \mean{\mathcal{N}_{#1}} }
\newcommand{\ncf} [ 1 ] { \mathcal{N}_{#1}' }
% generic vector/tensor
\newcommand{\vv}[ 1 ]{ \mathbf{#1} }
\newcommand{\xv}{ \vv{x} }
\newcommand{\yv}{ \vv{y} }
\newcommand{\te}[1]{ \mathbf{#1} }
\newcommand{\xt}{ \te{x} }
\newcommand{\yt}{ \te{y} }
% "order of" symbol
\newcommand{\order} [2] { 
    \MyIfEmptyTF{#2}
    {\textit{O}(#1) }
    {\textit{O}({#1}^{#2}) }
}
\newcommand{\code} [ 1 ] { \texttt{#1} }
% evaluate (eg derivative) at position
\newcommand{\atpos} [2] { \bigg|_{{#1} = {#2}} } % use evalfun instead!
\newcommand{\evalfunI} [3] {\left.\kern-\nulldelimiterspace {#1} \right|_{{#2} = {#3}}}
\newcommand{\evalfun} [3] {\left. #1 \right|_{{#2} = {#3}}}
% transformed (symmetries)
\newcommand{\tra} [1] {\traAltEq{#1}}
\newcommand{\traAnco} [1] {#1 \rightarrow}
\newcommand{\traAlt} [1] {#1^* }
\newcommand{\traAltEq} [1] {{#1}^* = }

\newcommand{\einsteinSum} [2][] {
    \iftoggle{EinsteinSumConvention}
        {\MyIfEmptyTF{#1}
            {}
            \else
            {#1}
            \fi}
        {\MyIfEmptyTF{#1}
            {\sum_{{#2}=1}^{3}}
            {\sum_{{#2}=1}^{3} \left( {#1} \right)}}}
% representation of similarity variables
\newcommand{\simvar} [1] {\widetilde{#1}}
% averaging
\ExplSyntaxOn
\newcommand{\mean} [1] {
    \tl_set:Nn \l_tmpa_tl { #1 } % store argument in variable \l_tmpa_tl
    \int_compare:nTF { \tl_count:N \l_tmpa_tl = 1 } % check if the argument only contains a single value
        { \bar{#1} } % if there is only one letter, use the bar macro
        { \overline{#1}} % otherwise use the longer overline
    }
% derivatives
\newcommand{\diffGeneric}[4][]{
    \MyIfEmptyTF{#1}
        { \cfrac{#4 #2}{#4 #3} }
        { \int_compare:nTF{ #1=2 } % second derivatives in Einstein notation
            {\iftoggle{EinsteinNotationDiff} % check if option is enabled
                {\cfrac{#4^{#1} #2}{#4 #3 #4 #3} }
                {\cfrac{#4^{#1} #2}{#4 #2^{#3}} }
            } 
            {{\cfrac{#4^{#1} #2}{#4 #3 ^{#1}}}} % higher derivatives, Einstein notation won't make sense in general
        }
}
\ExplSyntaxOff
% total derivative
\newcommand{\dd} [3][] { \diffGeneric[#1]{#2}{#3}{d} }
% partial derivative
\newcommand{\pd} [3][] { \diffGeneric[#1]{#2}{#3}{\partial} }
% material derivative
\newcommand{\md} [3][] { \diffGeneric[#1]{#2}{#3}{D} }
% derivatives without frac 
\newcommand{\diffInline} [3] { {#3 #1}/{#3 #2} }
\newcommand{\did} [2] { \diffInline{#1}{#2}{d} }
\newcommand{\pdi} [2] { \diffInline{#1}{#2}{\partial} }
\newcommand{\mdi} [2] { \diffInline{#1}{#2}{D} }
% jet derivative
% \newcommand{\jd} [2] { \underset{#2}{#1} }
% \newcommand{\jd} [2] { {#1}_{{\!,{#2}}} }
\newcommand{\jd} [2] { {#1}{}_{{\!,{#2}}} }

% for a long first arg it is nicer to write d/dx(z+longArg)
\newcommand{\diffLongGeneric} [4] {\diffGeneric{}{#2}{#3}{#4}{#1}}
\newcommand{\ddl} [3] { \diffLongGeneric{#1}{#2}{d}{#3} }
\newcommand{\pdl} [3] { \diffLongGeneric{#1}{#2}{\partial}{#3} }
\newcommand{\mdl} [3] { \diffLongGeneric{#1}{#2}{D}{#3} }

\newcommand{\diffSecond} [4] {
    \cfrac{{#1} ^2 {#2}}{{#1} {#3} {#1} {#4}}
}
% total derivative
\newcommand{\ddSecond} [3] { \diffSecond{d}{#1}{#2}{#3} }
% partial derivative
\newcommand{\pdSecond} [3] { \diffSecond{\partial}{#1}{#2}{#3} }
% material derivative
\newcommand{\mdSecond} [3] { \diffSecond{D}{#1}{#2}{#3} }


% velocity moments
\newcommand{\ui}[1]{U_{#1}} % unaveraged velocity
\newcommand{\uif}[1]{u_{#1}} % unaveraged fluctuating velocity

\newcommand{\umv}  {
\bar{\vv{U}}
}
\newcommand{\uav}  {
\hat{\vv{U}}
}
\newcommand{\ufv}  {
\mean{\vv{u} \otimes \vv{u}}
}

\newcommand{\ufFirst} [1] { u_{#1} }
\newcommand{\ufHigherShort} [1] {%
        {\iftoggle{OnePointExponent}
        {R^{(0)}_{#1}} 
        {R_{#1}}}
        }
\newcommand{\uftHigherShort} [1] {%
        {R_{#1}}
        }

\newcommand{\ufHigherLong} [1] {%
    \mean{\foreachletter{\uif}{}{#1}}
}

\newcommand{\ufHigher} [1] {%
        {\iftoggle{ShortHigherMoments}
        {\ufHigherShort{#1}} 
        {\ufHigherLong{#1}}}
        }

\newcommand{\uftHigher} [1] {%
        {\iftoggle{ShortHigherMoments}
        {\uftHigherShort{#1}} 
        {\ufHigherLong{#1}}}
        }

\newcommand{\umFirst} [1] { \mean{U}_{#1} }
\newcommand{\umHigherShort} [1] {%
        {\iftoggle{OnePointExponent}
        {H^{(0)}_{#1}} 
        {H_{#1}}}
        }

\newcommand{\umHigherLong} [1] {%
    \mean{\foreachletter{\ui}{}{#1}}
}

\newcommand{\umtHigherShort} [1] {%
        {H_{#1}}
        }

\newcommand{\umtHigherLong} [1] {%
    \mean{\foreachletter{\ui}{}{#1}}
}

\newcommand{\umHigher} [1] {%
        {\iftoggle{ShortHigherMoments}
        {\umHigherShort{#1}} 
        {\umHigherLong{#1}}}
        }

\newcommand{\umtHigher} [1] {%
        {\iftoggle{ShortHigherMoments}
        {\umtHigherShort{#1}} 
        {\umHigherLong{#1}}}
        }

\ExplSyntaxOn
\newcommand{\uf} [1] {
    \tl_set:Nn \l_tmpa_tl { #1 } % store argument in variable \l_tmpa_tl
    \int_compare:nTF { \tl_count:N \l_tmpa_tl = 1 } % check if the argument only contains a single value
        {\ufFirst{#1}} %
        {\ufHigher{#1}} %
    }
\newcommand{\uft} [1] {
    \tl_set:Nn \l_tmpa_tl { #1 } % store argument in variable \l_tmpa_tl
    \int_compare:nTF { \tl_count:N \l_tmpa_tl = 1 } % check if the argument only contains a single value
        {\ufFirst{#1}} %
        {\uftHigher{#1}} %
    }
\newcommand{\um} [1] {
    \tl_set:Nn \l_tmpa_tl { #1 } % store argument in variable \l_tmpa_tl
    \int_compare:nTF { \tl_count:N \l_tmpa_tl = 1 } % check if the argument only contains a single value
        {\umFirst{#1}} %
        {\umHigher{#1}} %
    }
\newcommand{\umt} [1] {
    \tl_set:Nn \l_tmpa_tl { #1 } % store argument in variable \l_tmpa_tl
    \int_compare:nTF { \tl_count:N \l_tmpa_tl = 1 } % check if the argument only contains a single value
        {\umFirst{#1}} %
        {\umtHigher{#1}} %
    }
\ExplSyntaxOff
\renewcommand{\pm}  {\mean{P}}
\newcommand{\pf}  {p}
\ExplSyntaxOn
\newcommand{\umTensor} [1] { 
    \int_compare:nTF { #1 = 1 } % 
    { \vv{\um} }
    { \vv{H^0} }
}
\newcommand{\ufTensor} [1] { 
    \int_compare:nTF { #1 = 1 } % 
    { \vv{\uf} }
    { \vv{R^0} }
}
\newcommand{\umTensorTwoPt} [1] { 
    \int_compare:nTF { #1 = 1 } % 
    { \vv{\um} }
    { \vv{H} }
}
\newcommand{\ufTensorTwoPt} [1] { 
    \int_compare:nTF { #1 = 1 } % 
    { \vv{\uf} }
    { \vv{R} }
}
\ExplSyntaxOff
\newcommand{\pum} [1] {
        \mean{PU_{#1}}^{(0)}
        % {\mean{PU_{#1}}(\vv{#2},\vv{#3})}
}
\newcommand{\upm} [1] {
        \mean{U_{#1}P}^{(0)}
        % {\mean{U_{#1}P}(\vv{#2},\vv{#3})}
}
\newcommand{\umpx} [2] {
    \mean{U_{#1} \jd{P}{x_{#2}}}^{(0)}
        % {\mean{U_{#1}P}(\vv{#2},\vv{#3})}
}
\newcommand{\puf} [1] {
        \mean{pu_{#1}}^{(0)}
        % {\mean{pu_{#1}}(\vv{#2},\vv{#3})}
}
\newcommand{\upf} [1] {
        \mean{u_{#1}p}^{(0)}
        % {\mean{u_{#1}p}(\vv{#2},\vv{#3})}
}
% two-point correlations
% instantaneous approach
% \newcommand{\umtWithoutArgsShort} [8] {
%     \MyIfEmptyTF{#1}
%             { \um }
%             {\MyIfEmptyTF{#2}
%                 % {\um{#1} }
%                 {\um{#1}(\vv{#5}) }% always show argument for single variable
%                 { \MyIfEmptyTF{#3} 
%                     {H_{#1 #2} }
%                     { \MyIfEmptyTF{#4}
%                         {H_{#1 #2 #3} }
%                     {H_{#1 #2 #3 #4} }
%                 }
%             }
%         }
%      }
% \newcommand{\umtWithArgsShort} [8] {
%         \MyIfEmptyTF{#1}
%             { \um(\vv{#5}) }
%             {\MyIfEmptyTF{#2}
%                 {\um{#1}(\vv{#5}) }
%                 { \MyIfEmptyTF{#3} 
%                     {H_{#1 #2}(\vv{#5},\vv{#6}) }
%                     { \MyIfEmptyTF{#4}
%                         {H_{#1 #2 #3}(\vv{#5}, \vv{#6}, \vv{#7}) }
%                         {H_{#1 #2 #3 #4}(\vv{#5}, \vv{#6}, \vv{#7}, \vv{#8}) }
%                 }
%             }
%         }
%  }
% \newcommand{\umtShort} [8] { 
%     \MyIfEmptyTF{#5}
%     { \umtWithoutArgsShort{#1}{#2}{#3}{#4}[#5][#6][#7][#8] }
%     { \iftoggle{ShowArgsTwoPoint}
%         { \umtWithArgsShort{#1}{#2}{#3}{#4}[#5][#6][#7][#8] }
%         { \umtWithoutArgsShort{#1}{#2}{#3}{#4}[#5][#6][#7][#8] }
%     }
% }
% \newcommand{\umtWithoutArgsLong} [8] {
%     \MyIfEmptyTF{#1}
%             { \um }
%             {\MyIfEmptyTF{#2}
%                 {\um{#1} }
%                 {\um{#1}(\vv{#5}) }% always show argument for single variable
%                 { \MyIfEmptyTF{#3} 
%                     { \mean{ U_{#1} U_{#2} } }
%                     { \MyIfEmptyTF{#4}
%                         { \mean{U_{#1} U_{#2} U_{#3}} }
%                         { \mean{U_{#1} U_{#2} U_{#3} U_{#4}} }
%                 }
%             }
%         }
%      }
% \newcommand{\umtWithArgsLong} [8] {
%         \MyIfEmptyTF{#1}
%             { \um(\vv{#5}) }
%             {\MyIfEmptyTF{#2}
%                 {\um{#1}(\vv{#5}) }
%                 { \MyIfEmptyTF{#3} 
%                     { \mean{ U_{#1}(\vv{#5}) U_{#2}(\vv{#6})} }
%                     { \MyIfEmptyTF{#4}
%                         { \mean{ U_{#1}(\vv{#5}) U_{#2}(\vv{#6}) U_{#3}(\vv{#7})} }
%                         { \mean{ U_{#1}(\vv{#5}) U_{#2}(\vv{#6}) U_{#3}(\vv{#7}) U_{#4}(\vv{#8})} }
%                 }
%             }
%         }
%  }
% \newcommand{\umtLong} [8] { 
%     \MyIfEmptyTF{#5}
%     { \umtWithoutArgsLong{#1}{#2}{#3}{#4}[#5][#6][#7][#8] }
%     { \iftoggle{ShowArgsTwoPoint}
%         { \umtWithArgsLong{#1}{#2}{#3}{#4}[#5][#6][#7][#8] }
%         { \umtWithoutArgsLong{#1}{#2}{#3}{#4}[#5][#6][#7][#8] }
%     }
% }
%  \newcommand{\umt} [8] { 
%      \iftoggle{ShortHigherMoments}
%      {\umtShort{#1}{#2}{#3}{#4}[#5][#6][#7][#8]}
%      {\umtLong{#1}{#2}{#3}{#4}[#5][#6][#7][#8]}
%  }
% % fluctuation approach
% \newcommand{\uftWithoutArgsShort} [4] {
%     \MyIfEmptyTF{#1}
%             { \uf }
%             {\MyIfEmptyTF{#2}
%                 {\uf{#1} }
%                 { \MyIfEmptyTF{#3} 
%                     {R_{#1 #2} }
%                     { \MyIfEmptyTF{#4}
%                         {R_{#1 #2 #3} }
%                     {R_{#1 #2 #3 #4} }
%                 }
%             }
%         }
%      }
% \newcommand{\uftWithArgsShort} [8] {
%         \MyIfEmptyTF{#1}
%             { \uf(\vv{#5}) }
%             {\MyIfEmptyTF{#2}
%                 {\uf{#1}(\vv{#5}) }
%                 { \MyIfEmptyTF{#3} 
%                     {R_{#1 #2}(\vv{#5},\vv{#6}) }
%                     { \MyIfEmptyTF{#4}
%                         {R_{#1 #2 #3}(\vv{#5}, \vv{#6}, \vv{#7}) }
%                         {R_{#1 #2 #3 #4}(\vv{#5}, \vv{#6}, \vv{#7}, \vv{#8}) }
%                 }
%             }
%         }
%  }
% \newcommand{\uftShort} [8] { 
%     \MyIfEmptyTF{#5}
%     { \uftWithoutArgsShort{#1}{#2}{#3}{#4} }
%     { \iftoggle{ShowArgsTwoPoint}
%         { \uftWithArgsShort{#1}{#2}{#3}{#4}[#5][#6][#7][#8] }
%         { \uftWithoutArgsShort{#1}{#2}{#3}{#4} }
%     }
% }
% \newcommand{\uftWithoutArgsLong} [4] {
%     \MyIfEmptyTF{#1}
%             { \uf }
%             {\MyIfEmptyTF{#2}
%                 {\uf{#1} }
%                 { \MyIfEmptyTF{#3} 
%                     { \mean{ u_{#1} u_{#2} } }
%                     { \MyIfEmptyTF{#4}
%                         { \mean{U_{#1} u_{#2} u_{#3}} }
%                         { \mean{U_{#1} u_{#2} u_{#3} u_{#4}} }
%                 }
%             }
%         }
%      }
% \newcommand{\uftWithArgsLong} [8] {
%         \MyIfEmptyTF{#1}
%             { \uf(\vv{#5}) }
%             {\MyIfEmptyTF{#2}
%                 {\uf{#1}(\vv{#5}) }
%                 { \MyIfEmptyTF{#3} 
%                     { \mean{ u_{#1}(\vv{#5}) u_{#2}(\vv{#6})} }
%                     { \MyIfEmptyTF{#4}
%                         { \mean{ u_{#1}(\vv{#5}) u_{#2}(\vv{#6}) u_{#3}(\vv{#7})} }
%                         { \mean{ u_{#1}(\vv{#5}) u_{#2}(\vv{#6}) u_{#3}(\vv{#7}) u_{#4}(\vv{#8})} }
%                 }
%             }
%         }
%  }
% \newcommand{\uftLong} [8] { 
%     \MyIfEmptyTF{#5}
%     { \uftWithoutArgsLong{#1}{#2}{#3}{#4} }
%     { \iftoggle{ShowArgsTwoPoint}
%         { \uftWithArgsLong{#1}{#2}{#3}{#4}[#5][#6][#7][#8] }
%         { \uftWithoutArgsLong{#1}{#2}{#3}{#4} }
%     }
% }
%  \newcommand{\uft} [8] { 
%      \iftoggle{ShortHigherMoments}
%      {\uftShort{#1}{#2}{#3}{#4}[#5][#6][#7][#8]}
%      {\uftLong{#1}{#2}{#3}{#4}[#5][#6][#7][#8]}
%  }
\newcommand{\puft} [2][] {
    \MyIfEmptyTF{#1}
        {\mean{pu_{#2}}}
        {\mean{pu_{#2}}}
        % {\mean{pu_{#1}}(\vv{#2},\vv{#3})}
}
\newcommand{\upft} [2][] {
    \MyIfEmptyTF{#1}
        {\mean{u_{#2}p}}
        {\mean{u_{#2}p}}
        % {\mean{u_{#1}p}(\vv{#2},\vv{#3})}
}
\newcommand{\pumt} [2][] {
    \MyIfEmptyTF{#1}
        {\mean{PU_{#2}}}
        {\mean{PU_{#2}}}
        % {\mean{PU_{#1}}(\vv{#2},\vv{#3})}
}
\newcommand{\upmt} [2][] {
    \MyIfEmptyTF{#1}
        {\mean{U_{#2}P}}
        {\mean{U_{#2}P}}
        % {\mean{U_{#1}P}(\vv{#2},\vv{#3})}
}
%LES
% \newcommand{\ufilter} [4] {
%     \MyIfEmptyTF{#1}
%         { \tilde{U} }
%         {\MyIfEmptyTF{#2}
%             {\tilde{U}_{#1}}
%             { \MyIfEmptyTF{#3} 
%                 {\tau_{#1 #2}}
%                 { \MyIfEmptyTF{#4}
%                     {\tau_{#1 #2 #3}}
%                 {H_{#1 #2 #3 #4}}
%             }
%         }
%     }
% }
\newcommand{\pfilter}  {\tilde{P}}
% turbulent viscosity
% \newcommand{\nut}  {\nu_{\text{turb}}}
\newcommand{\nut}  {\nu_{t}}
% artificial velocity 
\newcommand{\ua} [1][] { 
    \MyIfEmptyTF{#1}
    {\hat{U}} 
    {\hat{U}_{#1}} 
} 
\newcommand{\pa}  { \hat{P} }
\newcommand{\ka}  { \hat{k} }
\newcommand{\ualpha} [1][] { 
    \MyIfEmptyTF{#1}
        {v} 
        {v_{#1}} 
} 
% k- eps - model
\newcommand{\kepsmodel}  {{\(k\)-\(\varepsilon\)-model}}
% k- omega - model
\newcommand{\komegamodel}  {{\(k\)-\(\omega\)-model}}
%Symmetry generators
\newcommand{\infGen} [1][] {
    \MyIfEmptyTF{#1}
    {X}
        {X_{#1}}
    }
    \newcommand{\infGenT}  {\infGen[t]}
\newcommand{\infGenX} [1][] {
    \MyIfEmptyTF{#1} 
    {\infGen[x_i]}
    {\infGen[x_{#1}]}
}
\newcommand{\infGenGal} [1][] {
    \MyIfEmptyTF{#1} 
    {\infGen[\text{Gal}]}
    {\infGen[{\text{Gal}_{{#1}}}]}
}
\newcommand{\infGenGalDiff} [1][] {
    \MyIfEmptyTF{#1} 
    {\tilde{\infGen}_{\text{Gal}}}
    {\tilde{\infGen}_{{\text{Gal}_{{#1}}}}}
}
\newcommand{\infGenRot} [1][]
{\MyIfEmptyTF{#1}
    {\infGen[{\text{rot}_{\alpha}}]}
    {\infGen[{\text{rot}_{#1}}]}}

\newcommand{\infGenP}  {\infGen[P]}
\newcommand{\infGenPDiff}  {\tilde{\infGen}_[P]}
\newcommand{\infGenScI}  {\infGen[\text{Sc},I]}
\newcommand{\infGenScII}  {\infGen[\text{Sc},II]}
\newcommand{\infGenScNs}  {\infGen[\text{Sc,ns}]}
\newcommand{\infGenScStat}  {\infGen[\text{Sc,stat}]}
\newcommand{\infGenTransStat} [1] {X_{{\text{Tr,stat,{#1}}}}}
\newcommand{\infGenTransStatI} [1][] {%
    \MyIfEmptyTF{#1}
        {X_{\text{Tr,stat},I}}
        {X_{\text{Tr,stat},I, {#1} }}}
\newcommand{\infGenTransStatII} [1][] {%
    \MyIfEmptyTF{#1}
        {X_{\text{Tr,stat},II}}
        {X_{\text{Tr,stat},II, {#1} }}}
%Global form
\newcommand{\gloSymm} [1][] {
    \MyIfEmptyTF{#1}
    {T}
    {T_{#1}}
}
\newcommand\gloSymmT{\gloSymm[t]}
\newcommand{\gloSymmX} [1][] {
    \MyIfEmptyTF{#1}
        {\gloSymm[x_i]}
        {\gloSymm[x_{#1}]}
}
\newcommand{\gloSymmGal} [1][] {
    \MyIfEmptyTF{#1}
        {\gloSymm[\text{Gal}_{i}]}
        {\gloSymm[\text{Gal}_{#1}]}
}
\newcommand{\gloSymmRot}  {\gloSymm[\text{rot}_{\alpha}]}
\newcommand{\gloSymmP}  {\gloSymm[P]}
\newcommand{\gloSymmScI}  {\gloSymm[\text{Sc},I]}
\newcommand{\gloSymmScII}  {\gloSymm[\text{Sc},II]}
\newcommand{\gloSymmScNs}  {\gloSymm[\text{Sc,ns}]}
\newcommand{\gloSymmScStat}  {\gloSymm[\text{Sc,stat}]}
\newcommand{\gloSymmTransStat} [1][] {T_{\text{Tr,stat,{#1}}}}
\newcommand{\gloSymmTransStatI} [1][] {%
    \MyIfEmptyTF{#1}
        {T_{\text{Tr,stat},I}}
        {T_{\text{Tr,stat},I, {#1} }}}
\newcommand{\gloSymmTransStatII} [1][] {%
    \MyIfEmptyTF{#1}
        {T_{\text{Tr,stat},II}}
        {T_{\text{Tr,stat},II, {#1} }}}

\newcommand{\lagX} [1] {\xi_{#1}}
% group parameter
\newcommand{\grp}[1][]{ 
    \MyIfEmptyTF{#1}
    {a}
    {a_{#1}}
}
\newcommand{\grpT}  {\grp[T]}
\newcommand{\grpX} [1][] {
    \MyIfEmptyTF{#1}
        {\grp[x_i]}
        {\grp[x_{#1}]}
}
\newcommand{\grpGal} [1][] {
    \MyIfEmptyTF{#1}
        {\grp[\text{Gal}_{i}]}
        {\grp[\text{Gal}_{#1}]}
}
\newcommand{\grpRot} [1][] {
    \MyIfEmptyTF{#1}
        {\grp[\text{rot}_{\alpha}]}
        {\grp[\text{rot}_{#1}]}
}
\newcommand{\grpP}  {\grp[P]}
\newcommand{\grpScI}  {\grp[\text{Sc},I]}
\newcommand{\grpScII}  {\grp[\text{Sc},II]}
\newcommand{\grpScNs}  {\grp[\text{Sc,ns}]}
\newcommand{\grpScStat}  {\grp[\text{Sc,stat}]}
\newcommand{\grpTransStat} [1] {\grp[\text{Tr,stat,{#1}}]}
\newcommand{\grpTransStatI} [1][] {%
    \MyIfEmptyTF{#1}
        {\grp[\text{Tr,stat},I]}
        {\grp[\text{Tr,stat},I, {#1} ]}}
\newcommand{\grpTransStatII} [1][] {%
    \MyIfEmptyTF{#1}
        {\grp[\text{Tr,stat},II]}
        {\grp[\text{Tr,stat},II, {#1} ]}}
\newcommand{\grpTransStatIII} [1][] {%
    \MyIfEmptyTF{#1}
        {\grp[\text{Tr,stat},III]}
        {\grp[\text{Tr,stat},III, {#1} ]}}

% \newcommand{ \expgrp }[1][] { 
%     \iftoggle{GroupExpSimplified}
%     {\grp}
%     {e^{{#1} \grp}}
% }
%for galileian group
\newcommand{\ftGal} [1] {f_{\text{Gal}_{#1}} (t)}
\newcommand{\ftpGal} [1] {f_{\text{Gal}_{#1}}^{\prime} (t)}
\newcommand{\ftppGal} [1] {f_{\text{Gal}_{#1}}^{\prime\prime} (t)}
\newcommand{\ftpppGal} [1] {f_{\text{Gal}_{#1}}^{\prime\prime\prime} (t)}
%for pressure translation
\newcommand{\ftP}  {f_{P} (t)}
\newcommand{\ftpP}  {f_{P}^{\prime} (t)}
\newcommand{\ftppP}  {f_{P}^{\prime\prime} (t)}
% model for a term
\newcommand{ \model } [1] {\mathcal{M}_{#1}}

% integration constant
\newcommand{\intconst}[2][]{%
\MyIfEmptyTF{#1}
  {c_{#2}}
  {c_{{#2}_{#1}}}
}

% jet stuff
\newcommand{ \ucenterline }  {U_{s}}

% cite author's name in possesive form
% \newcommand{ \posscite } [1] { \citeauthor }
\newcommand{\crefrangeconjunction}{--}
